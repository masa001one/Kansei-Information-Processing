\documentclass[a4j]{ujarticle}
\西暦
\usepackage{graphicx}
\usepackage{url}
\usepackage{listings,jlisting}
\usepackage{ascmac}
\usepackage{amsmath,amssymb}

%ここからソースコードの表示に関する設定
\lstset{
  basicstyle={\ttfamily},
  identifierstyle={\small},
  commentstyle={\smallitshape},
  keywordstyle={\small\bfseries},
  ndkeywordstyle={\small},
  stringstyle={\small\ttfamily},
  frame={tb},
  breaklines=true,
  columns=[l]{fullflexible},
  numbers=left,
  xrightmargin=0zw,
  xleftmargin=3zw,
  numberstyle={\scriptsize},
  stepnumber=1,
  numbersep=1zw,
  lineskip=-0.5ex
}
%ここまでソースコードの表示に関する設定

\title{知能プログラミング演習II 課題0}
\author{
 3011XXXX 名工 大輔\\
  {\small (学生番号と氏名が必要)}
}
\date{\today}


\begin{document}
\maketitle

\paragraph{提出レポート: } rep0
\paragraph{自己評価: } A
\paragraph{評価の理由: } calcFiboSeqを実装するにあたり,XXXがYYYになるような独自の手法を工夫した.また,考案した独自手法がどうしてZZZのときにうまくいかないのかを考察し,改善策のアイデアまで示せたので.



\section{課題の説明}
\begin{description}
\item[課題0-1] N番目のフィボナッチ数をもとめるプログラムを実装せよ.
\item[課題0-2] 自然界に現れるフィボナッチ数列の例を示し,なぜそれがフィボナッチ数列になるのかという原理を考察せよ.
\end{description}


\section{課題0-1}
\begin{screen}
  N番目のフィボナッチ数をもとめるプログラムを実装せよ.
\end{screen}

私の担当箇所は、後述するcalcFiboメソッドの実装である。
\subsection{手法}
課題に加えて、以下の3点を独自仕様として組み込んだ。

\begin{enumerate}
\item 与えられた数がフィボナッチ数かどうかを判定する。
\item N番目のフィボナッチ数だけではなく,N番目までのフィボナッチ数列を返す。
\item フィボナッチ数の一般項を用いてN番目のフィボナッチ数を求める。
\end{enumerate}

1.に関して、ユーザーから数字が与えられた時、それがフィボナッチ数である時に true を返し、フィボナッチ数ではない時には false を返す仕様とした。私はこのcalcFiboの実装を担当した。

2.に関しては、整数Nが与えられた時、0番目からN番目までのフィボナッチ数列を配列として返す仕様とした。

3.に関しては、・・・
\subsection{実装}

まず、プログラムに含まれるクラスは以下の1つ。
\begin{itemize}
\item Fibonacciクラス: メソッドcalcFibo, isFibo, calcFiboSeq, calcFiboGeneral を実装したクラス.Fibonacci.java に含まれる。
\end{itemize}

n番目のフィボナッチ数を計算するcalcFiboメソッドの実装をソースコード\ref{src:calcFibo}に示す。

\begin{lstlisting}[caption=calcFiboメソッド,label=src:calcFibo]
    // n番目のフィボナッチ数を計算して返す
    int calcFibo(int n) {
	if (n <= 1) {
	    return 1;
	}
	return calcFibo(n - 1) + calcFibo(n - 2);
    }
\end{lstlisting}

isFiboメソッドおよびcalcFiboSeqの実装はXXXをXXXのように実装した。
特に,calcFiboSeq を実装するにあたり,XXXがYYYになるようにするために,XXXXに着目した独自手法を考案した.

calcFiboメソッドに渡す引数nをプログラム実行時に指定できるようにするため、mainメソッドおよびFibonacciクラスのコンストラクタを以下のように実装した。
\begin{lstlisting}[caption=mainメソッドとコンストラクタ,label=src:main]
    public static void main(String[] args) {
	new Fibonacci(Integer.parseInt(args[0]));
    }

    public Fibonacci(int n) {
	System.out.println(calcFibo(n));
    }
\end{lstlisting}

\subsection{実行例}
Fibonacciクラスに引数5を指定した実行結果を以下に示す。

\begin{lstlisting}
ckv14XX0@cse:~/eclipse-workspace/Fibo > java Fibonacci 5
8
\end{lstlisting}

今回は0, 1番目のフィボナッチ数を1としたため、m図で表したように calcFibo() の引数に5を与えた時は8となるのが正しい動作である。これは、・・・
\subsection{考察}
今回、フィボナッチ数をもとめるためのクラス Fibonacci を実装し、インスタンスメソッドとして機能を実装したが、これは静的クラスまたはシングルトンとして実装した方が適していたように考えられる。なぜなら・・・

また、calcFiboメソッドを再帰関数として実装するのではなく、イテレータを用いて実装した場合を考える。イテレータで実装した場合は、実行速度が・・・

calcFiboSeq を実装するにあたり,XXXがYYYになるようにするためにXXXXに着目したが,ZZZのときに処理が失敗するという問題が発生した.これは,ZZZがXXXに対応していないことが原因と考えられる.よって,ZZZを...することで改善できる可能性がある.

\section{課題0-2}
\begin{screen}
  自然界に現れるフィボナッチ数列の例を示し,なぜそれがフィボナッチ数列になるのかという原理を考察せよ.
\end{screen}

課題0-2は実装を伴わない課題であるため、考察のみ記す。

\subsection{考察}
文献\cite{kijima2012}によると、ひまわりの種の並び方にフィボナッチ数列が隠れているという。具体的には、・・・。その原理は、・・・であると考えられる。何故なら、図\ref{fig:himawari} \cite{notty} に示すように、・・・

\begin{figure}[!hbt]
  \centering
  \includegraphics[bb=0 0 782 352,width=0.6\linewidth]{himawari.png}
  \caption{ひまわりの種の並び方}
  \label{fig:himawari}
\end{figure}


\section{感想}
フィボナッチ数の一般項をもとめるのに夢中であやうく実装が間に合わなくなるところだった。実装に関しては、・・・


% 参考文献
\begin{thebibliography}{99}
\bibitem{kijima2012} 来嶋大二: ひまわりの螺旋, 数学のかんどころシリーズ 8, 共立出版, 2012.
\bibitem{notty} ひまわりに隠されたフィボナッチ数列と黄金比 – ひまわりは黄金の花?, 数学の面白いこと・役に立つことをまとめたサイト, \url{https://analytics-notty.tech/fibonacci-and-goldenratio-in-sunflower/} (2018年9月30日アクセス).

\bibitem{hanako} 工大花子さんのレポート。また、・・・を教えてもらった 

\end{thebibliography}

\end{document}
